%导言区
\documentclass[12pt]{article}
\usepackage[fontset=mac]{ctex}
\usepackage{xcolor}
\usepackage{CJK}
\xeCJKsetup{underline={format = \color{black}, thickness=1pt}}
\usepackage{graphicx}
\usepackage{fancyhdr}
\bibliographystyle{unsrt}
\usepackage{listings}
\lstset{                  %设置代码块
	basicstyle=\footnotesize\ttfamily,% 基本风格
	numbers=left,    % 行号
	numbersep=10pt,  % 行号间隔 
	tabsize=4,       % 缩进
	extendedchars=true, % 扩展符号
	backgroundcolor=\color[RGB]{245,245,244}, % 设定背景颜色
	keywordstyle=\color{blue}, % 设定关键字颜色
	numberstyle=\footnotesize\color{gray},  % 设定行号格式
	commentstyle=\it\color[RGB]{100, 200, 150}, % 设置代码注释的格式
	stringstyle=\color{red},
	breaklines=true, % 自动换行
	language=python,
	frame=leftline,  % 框架左边竖线
	xleftmargin=30pt,% 竖线左边间距
	showspaces=false,% 空格字符加下划线
	showstringspaces=false,% 字符串中的空格加下划线
	showtabs=false,  % 字符串中的tab加下划线
}
\setmainfont{Times New Roman}
\usepackage{titlesec}
\titleformat{\section}{\Large \bfseries}{\arabic{section}}{1em}{}
\titlespacing{\section}{0pt}{3.5ex plus 1ex minus .2ex}{2.3ex plus .2ex}

%\CTEXsetup[format={\raggedright \zihao{-3} \heiti},aftername={\qquad}]{section}
\usepackage{geometry}
\geometry{
	a4paper,
	left=3.17cm,
	right=3.17cm,
	top=2.54cm,
	bottom=2.54cm,
	headsep=1.5cm,
	footskip=1.75cm
}

%定义封面
\newcommand{\fillinblank}[2]{\underline{\makebox[#1]{#2}}}
\newcommand{\makecover}{
	\thispagestyle{empty}
	\begin{figure*}[htbp]
		\centering
		\includegraphics[height=3cm, width=13cm]{ucas_logo.pdf}
	\end{figure*}
	
	\vspace{20pt}
	\begin{center}
		{\heiti \zihao{1} \textbf{课程作业报告}}
	\end{center}
	\vspace{50pt}
	
	\begin{center}
		{		
			\fillinblank{33em}{\zihao{3} \heiti \textbf{基于卷积神经网络的数字手写体识别}}
		}
	\end{center}
	\vspace{80pt}
	
	\begin{flushleft}
		{
			\zihao{4}
			\renewcommand\arraystretch{1.5}
			\begin{tabular}{l@{\extracolsep{0.2em}}c}
				{\songti \textbf{作者姓名:}}    & \fillinblank{23em}{\textbf{李杨}}\\
				{\songti \textbf{学科专业:}}     & \fillinblank{23em}{\textbf{模式识别与智能系统}}\\
				{\songti \textbf{所在单位:}}    & \fillinblank{23em}{\textbf{中国科学院大学人工智能学院}}
			\end{tabular}
		}
	\end{flushleft}
	
	\vspace{100pt}
	\begin{center}
		{
			\zihao{4}
			\textbf{2020年5月}
		}
	\end{center}	
}

%定制章节格式


%定义页眉
\renewcommand{\headrule}{
	\vspace{10pt}
	\hrule width\headwidth height1.6pt \vspace{1.2pt}
	\hrule width\headwidth height0.4pt
}

\newcommand{\thisispagestyle}[3]{
	\pagestyle{fancy}
	\fancyhf{}
	\fancyhead[C]{\zihao{#1}\songti #2}
	\fancyhead[LE, RO]{\zihao{#1}\songti #3}
	\fancyfoot[C]{\thepage}
	%\renewcommand{\headrulewidth}{0pt} 
	%\renewcommand{\footrulewidth}{0pt}
}


%正文区
\begin{document}
   % \tableofcontents
    \makecover
    \newpage
    %\thisispagestyle{-4}{深度学习课程作业报告}{Lee}
    
    \setcounter{page}{1}
    \pagenumbering{arabic}
    

    \section*{摘要}
    这部分请用高度概括的语言说明该课程作业所研究问题\cite{ShafieeFast},采取的解决手段,已经解决该问题自己提出的创新或者改进(如果有的话),最后用一句话说明一下所完成项目达到的性能。

    \section{介绍}
    该部分可以依次介绍如下内容

    (1) 所要解决的问题,问题描述(体现其问题或者难点,该难点与后边自己提出的方案前后呼应是最好的);
    
    MNIST数据集是修改后的国家标准与技术研究所数据集的缩写\cite{redmon2016you}。它是一个由60000个28×28像素的小正方形灰度图像组成的数据集,这些图像的手写单位数介于0和9之间。任务是将手写数字的给定图像分类为10个类中的一个,这些类表示0到9之间的整数值(包括0和9)。
    
    它是一个广泛使用和深入理解的数据集,并且在大多数情况下是“已解决”的。性能最好的模型是深度学习卷积神经网络\cite{lin2017feature},其分类精度达到99\%以上,保留测试数据集的错误率在0.4\%到0.2\%之间。

    (2) 为了解决该问题对该领域所做的调研,比如目前该领域对该问题的代表性工作;

    (3) 概括介绍下自己提出的解决方案或者贡献点。
    
    \section{解决方案}
    \label{solution}
    该部分内容要条例清晰的介绍下所提出的解决方法\cite{papageorgiou1998general},包含损失函数设计、网络结构设计,创新点(如果有自己的创新贡献的话);

    \section{实验结果与分析}
    该部分首先介绍实验配置,包含数据集介绍,实验环境,例如选择什么深度学习平台,以及比较算法(如果想做比较的话),训练过程等;

    接着,给出算法整体性能,表格或者图表,配合文字解释和分析;
    
    最后,希望能提供相关的ablation study的分析,比如对于不同参数怎么选择,不同网络层数或者损失函数中不同正则项的有效性分析等;
    
    为了估计一个模型在一般问题上的性能,我们可以使用k倍交叉验证,或者5倍交叉验证。这将在训练和测试数据集的差异以及学习算法的随机性方面,给出一些模型的方差。考虑到标准差,模型的性能可以作为k-折叠的平均性能,如果需要,可以用它来估计置信区间。
    
    该模型主要有两个部分:前端特征提取由卷积层和池化层组成,后端分类器进行预测。
    
    对于卷积前端,我们可以从单个卷积层开始,该卷积层具有较小的过滤器大小(3,3)和少量的过滤器(32),然后是最大池化层。然后可以展平过滤器映射,为分类器提供特性。
    
    考虑到该问题是一个多类分类任务,我们知道我们需要一个具有10个节点的输出层来预测属于这10个类中每个类的图像的概率分布。这还需要使用SoftMax激活功能。在特性提取器和输出层之间,我们可以添加一个全连接层来解释特性,在本例中是100个节点。
    
    所有层都将使用relu激活函数和He 权重初始化方案,这两个都是最佳方法。
    
    我们将对学习率为0.01,动量为0.9的随机梯度下降优化器使用保守配置。分类交叉熵损失函数将得到优化,适用于多类分类,我们将监测分类精度指标,这是适当的,因为我们在10个类中的每一类都有相同数量的例子。
    
    模型将通过五重交叉验证进行评估。选择k=5的值为重复评估提供基线,并且不需要太长的运行时间。每个测试集将是训练数据集的20%,或大约12000个示例,接近此问题的实际测试集大小。
    
    训练数据集在分割前进行洗牌,每次都进行样本洗牌,这样我们评估的任何模型在每个折叠中都将具有相同的训练和测试数据集,从而提供模型之间的逐个比较。
    
    我们将为一个适度的10个训练阶段培训基线模型,默认批量大小为32个示例。每个阶段的测试集将用于评估模型在训练运行的每个阶段,以便我们可以稍后创建学习曲线,并在运行结束时,以便我们可以评估模型的性能。因此,我们将跟踪每次运行的结果历史,以及折叠的分类精度。
    
    \begin{lstlisting}[language=python] 
    #!/usr/bin/env python
    print('Hello world!')
    \end{lstlisting}

    \section{结论}
    这部分概括总结下自己所做的项目。
    
    在这个教程中,您学会了如何从头开始为手写数字分类开发卷积神经网络。
    具体来说,你学到了:
    
    如何开发测试工具以开发对模型的稳健评估并为分类任务建立性能基线。
    如何探索基线模型的扩展,以提高学习和模型容量。
    如何开发最终模型,评估最终模型的性能,并使用它来预测新图形
    
    \newpage
    \bibliography{refs}
\end{document}