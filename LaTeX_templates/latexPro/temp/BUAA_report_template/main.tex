%!TeX program = xelatex
\documentclass[12pt,hyperref,a4paper,UTF8]{ctexart}
\usepackage{BUAAReport}
\usepackage{listings}
\usepackage{xcolor}

\usepackage{setspace}
\setstretch{1.5} % 设置全局行距为1.5倍

\usepackage{enumitem} % 载入enumitem包以便自定义列表环境
\setlist[itemize]{itemsep=0pt, parsep=0pt} % 设置itemize环境的项目间距和段落间距

\setmainfont{Times New Roman} % 英文正文为Times New Roman

%封面页设置
{   
    %标题
    \title{ 
        \vspace{1cm}
        \heiti \Huge \textbf{{XXXX课程报告}} \par
        \vspace{1cm} 
        \heiti \Large {\underline{XXXXXX进展调研}}    
        \vspace{3cm}
    }

    \author{
        \vspace{0.5cm}
        \kaishu\Large 学院\ \dlmu[9cm]{计算机学院} \\ %学院
        \vspace{0.5cm}
        \kaishu\Large 专业\ \dlmu[9cm]{计算机科学与技术} \\ %班级
        \vspace{0.5cm}
        \kaishu\Large 学号\ \dlmu[9cm]{2023XXXXXX} \qquad  \\ %学号
        \vspace{0.5cm}
        \kaishu\Large 姓名\ \dlmu[9cm]{XXX} \qquad \\ %姓名 
    }
        
    \date{\today} % 默认为今天的日期,可以注释掉不显示日期
}
%%------------------------document环境开始------------------------%%
\begin{document}

% %%-----------------------封面--------------------%%
% \cover

% %%------------------摘要-------------%%
% %\begin{abstract}
% %
% %在此填写摘要内容
% %
% %\end{abstract}

% \thispagestyle{empty} % 首页不显示页码

% %%--------------------------目录页------------------------%%
% % \newpage
% % \tableofcontents
% % \thispagestyle{empty} % 目录不显示页码

% %%------------------------正文页从这里开始-------------------%
% \newpage
% \setcounter{page}{1} % 让页码从正文开始编号

% %%可选择这里也放一个标题
% %\begin{center}
% %    \title{ \Huge \textbf{{标题}}}
% %\end{center}

% \section{模板说明}
% 本模板主要适用于一些课程的平时论文以及期末论文,默认页边距为2.54cm和3.18cm,中文宋体,英文Times New Roman,字号为12pt(小四)。

% 编译方式:\verb|xelatex -> bibtex -> xelatex*2|


% 默认模板文件由以下四部分组成:
% \begin{itemize}
%     \item \texttt{main.tex} 主文件
%     \item \texttt{reference.bib} 参考文献,使用bibtex
%     \item \texttt{BUAAReport.sty} 文档格式控制,包括一些基础的设置,如页眉、标题、学院、学号、姓名等
%     \item \texttt{figures} 放置图片的文件夹
% \end{itemize}

% 第一次使用时需前往\texttt{BUAAReportReport.sty} 对标题、姓名、学号、页眉等进行设置,设置完后即可一劳永逸,封面LOGO亦可替换。

% 默认带有封面页,页码从正文开始。

% \section{一些插入功能}
% \subsection{插入公式}
% 行内公式$v-\varepsilon+\phi=2$。

% 插入行间公式如\autoref{Euler}:
% \begin{equation}
%     v-\varepsilon+\phi=2
%     \label{Euler}
% \end{equation}

% % \subsection{插入图片}
% % BUAA校徽如\autoref{BUAA}所示,注意这里使用了\verb|~\autoref{}|命令,也就是会自动生成“图”“式”等前缀,无需手动输入。

% % \begin{figure}[!htbp]
% %     \centering
% %     \includegraphics[width =.7\textwidth]{figures/buaa_logoname.eps}
% %     \caption{北京航空航天大学}
% %     \label{BUAA}
% % \end{figure}

% 插入上面图片的代码:

% % \begin{verbatim}
% % \begin{figure}[!htbp]
% %     \centering
% %     \includegraphics[width =.8\textwidth]{figures/buaa_logoname.eps}
% %     \caption{北京航空航天大学}
% %     \label{BUAA}
% % \end{figure}
% % \end{verbatim}

% \subsection{插入文本框}
% 本模板定义了一个圆角灰底的文本框,使用简化命令\verb|\tbox{}|即可,如果你不喜欢,可以前往 \texttt{BUAAReport.sty}对其进行修改。

% \tbox{
%     这是一个圆角灰底的文本框
% }

% \subsection{插入表格}
% 本模板文件如表~\ref{doc} 所示。
% \begin{table}[!htbp]
%     \centering
%     \begin{tabular}{l  | l}
%     \hline
%         文件名 & 说明 \\
%         \hline
%         \texttt{main.tex}  & 主文件 \\
%         \texttt{reference.bib} & 参考文献 \\
%         \texttt{BUAAReport.sty}  & 文档格式控制\\
%         \texttt{figures}  & 图片文件夹 \\
%         \hline
%     \end{tabular}
%     \caption{本模板文件组成}
%     \label{doc}
% \end{table}

% %\section{定理环境}
% %\begin{Theorem}
% %\end{Theorem}
% %
% %\begin{Lemma}
% %\end{Lemma}
% %
% %\begin{Corollary}
% %\end{Corollary}
% %
% %\begin{Proposition}
% %\end{Proposition}
% %
% %\begin{Definition}
% %\end{Definition}
% %
% %\begin{Example}
% %\end{Example}
% %
% %\begin{proof}
% %\end{proof}

% \subsection{插入高亮代码块}
% 利用\verb|lstlisting| 配置
% \begin{lstlisting}[style=CPP, title="c++代码"]
% #include <iostream>
% #include <array>
% int main()
% {
%     constexpr int MAX = 100;
%     std::array<int, MAX> arr;
% }  
% \end{lstlisting}

% \begin{lstlisting}[style=Java, title="Java代码"]
% public void addAdvertisement(String company, String ad_Category, String ad_Type, String ad_Price)
% {
%     int price = Integer.parseInt(ad_Price);
%     ad = new Advertisement(company, ad_Category, ad_Type, price);
%     adList.add(index, ad);
%     index++;
%     anDM = getDefaultDirectoryManager();
%     ActorTuple tuple = new ActorTuple(getActorName(), "advertiser",
%     company, ad_Category, ad_Type, price, index-1);
%     send(anDM, "register", tuple);
% }
% \end{lstlisting}

% \begin{lstlisting}[style=Python, title="Python代码"]                
% import random
% import collections
% Card = collections.namedtuple('Card', ['rank', 'suit'])

% class FrenchDesk:
%     ranks = [str(n) for n in range(2, 11)] + list('JQKA')
%     suits = 'spades diamonds clubs hearts'.split()
    
%     def __init__(self):
%         self._cards = [Card(rank, suit) for rank in self.ranks for suit in self.suits]
        
%     def __len__(self):
%         return len(self._cards)
        
%     def __getitem__(self, position):
%         return self._cards[position]
% desk = FrenchDesk()
% \end{lstlisting}

% \subsection{插入参考文献}
% 直接使用\verb|\cite{}|即可\cite{DBLP:conf/nips/VaswaniSPUJGKP17}。

% 例如:


   \textit{ 此处引用了文献\cite{0Isaac}。此处引用了文献\cite{2016The}}


% 引用过的文献会自动出现在参考文献中。


% %%----------- 参考文献 -------------------%%
% %在reference.bib文件中填写参考文献,此处自动生成

\reference


\end{document}